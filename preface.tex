The following Two-part Exercises are intended for the use of 
singing-classes ; the plan on which they are constructed being, 
it is thought, unique. The Author, after a lengthened experience 
in class-teaching, believes it to be the plan best adapted to pro- 
duce good, independent, readers of vocal music. A sight-singer, to 
be thorough, must be able to sing with readiness in all keys and 
in all sorts of time ; and in order to attain this readiness he must 
have practice in scale and time exercises, of such a character as 
can be mastered by one whose familiarity with the lines and 
spaces has to be gradually acquired. Now the way in which 
scale-exercises are usually presented to the members of singing- 
classes is so unattractive, that large numbers of students soon 
grow tired of practising them ; while the plan of exercising pupils 
in beating time without any connection with tune, although 
perhaps necessary at the very outset, soon becomes irksome. But 
by combining time with tune, and the scales with such contra- 
puntal devices as are to be found in the following pages, the 
interest of the pupils is sustained, the greatest possible amount 
of useful practice is obtained at the smallest cost of tediousness ; 
practice and pleasure go hand in hand, and progress is the result. 
In using this book teachers may adopt any system of solmiza- 
tion they may deem best, and also their own methods of explaining 
time and key-signatures, the relative durations of notes, &c. 
One direction, however, is necessary. The teacher must first 
teach the whole of his class to sing a scale. Then the scale in 
time. Then he must divide the class into two sections. Then 
direct those on his right hand to sing the scale, and those on 
his left the counterpoint. Then get those who before sang the 
counterpoint to sing the scale, and vice versa, and keep them 
reversing in this manner until each exercise has been completely 
mastered. The Author believes that no other plan can be 
adopted by which the habit of singing in time can be so well 
acquired ; because by the plan recommended each half of the 
class will in its turn be doing that which will be well within its 
power, viz., singing a scale, and thus giving support to those 
members of the class who are practising the counterpoint. The 
support thus afforded will be much more valuable for ultimate 
purposes than that which can be obtained by the strumming of 
every note on a pianoforte ; as it will be the support resulting 
from the harmonic relationship suggested to the minds of the 
pupils, and not merely that of sounds to be imitated. 

Bristol, Aug. 28th, 1883. 

